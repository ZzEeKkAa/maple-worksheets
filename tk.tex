\documentclass{article}
\usepackage[ukrainian,english]{babel}
\usepackage[utf8]{inputenc}
\usepackage[T2A]{fontenc}
\usepackage{amsmath}

\title{Експериментальна робота №5 \\
з курсу „Теорiя керування” \\
студента 3-го курсу групи ОМ \\
Гаврилка Євгенія}
\date{2017-05-26}
%\author{Гаврилко Євгеній}
\begin{document}

\maketitle
\newpage
\section{Опис математичної моделi i постановка задачi}
\subsection{Модель бойових дiй двох армiй}
Нехай в протиборствi беруть участь як регулярнi армiї, так i парти-
занськi об’єднання. Головною характеристикою суперникiв є чисельнiсть
сторiн $N_1(t) \geq 0$, $N_2(t) \geq 0$. У випадку дiй мiж регулярними частинами
динамiка їх чисельностi визначається факторами:
1). Швидкiсть зменшення особового складу за причинами, не зв’я-
заними безпосередньо з бойовими дiями: хвороби, травми, дезертирство;
2). Темп втрат обумовлений бойовими дiями; 3). Швидкiсть надання пiд-
крiплення, що вважається деякою функцiєю вiд часу.
При цих припущеннях для $N_1 (t)$, $N_2 (t)$ отримуємо систему диферен-
цiальних рiвнянь

\begin{equation}
{\frac {\rm d}{{\rm d}t}}N_1(t) = -a_1(t) N_1(t) - b_2(t) N_2(t) + g_1(t)
\end{equation}
\begin{equation}
{\frac {\rm d}{{\rm d}t}}N_2(t) = -b_1(t) N_1(t) - a_2(t) N_2(t) + g_2(t)
\end{equation}

Тут неперервнi функцiї $a_i(t) \geq  0$ визначають швидкiсть втрат у силу
причин, що не пов’язанi з бойовими дiями, неперервнi функцiї $b_i(t) \geq  0$
показують темпи втрат через бойовi дiї супротивника, неперервнi фун-
кцiї $g_i(t)$ показують темп надання пiдкрiплення (модель Ланчестера).

\subsection{Постановка задачі}
Розглядається система керування (1), (2), в якiй $g_1 (t), g_2(t)$ - функцiї керування,
$t \in [0, T]$, $a_1 (t) = a_1$, $a_2(t) = a_2$, $b_1 (t) = b_1$, $b_2 (t) = b_2$ - додатнi константи.

Побудувати множину досяжностi в момент $T > 0$ за умови, що обме-
ження на початковий стан задається нерiвностями

\begin{equation}
0\leq N_1\leq \alpha 
\end{equation}
\begin{equation}
0\leq N_2\leq \beta
\end{equation}
а обмеження на керування мають вигляд
\begin{equation}
-h_1\leq g_1(t)\leq h_1
\end{equation}
\begin{equation}
-h_2\leq g_2(t)\leq h_2
\end{equation}
де $\alpha, \beta, h1, h2$ – додатнi константи.
\section{Математичні викладки}

З теорії відомо, що система (1), (2) за умов (5), (6) є стандарною задачею множини досяжностi лiнiйної системи керування, яка може бути обчислена за формулою

\begin{equation}
X(t, M_0) = \Theta (t, t_0) M_0 +\int_{t_0}^t \Theta (t, s) B(s)U(s)\mathrm{d}s
\end{equation}
 
\begin{equation}
{\frac {\rm d}{{\rm d}t}}x \left( t \right) =-a_{{1}}x
 \left( t \right) -b_{{2}}y \left( t \right)
\end{equation}
\begin{equation}
 {\frac {\rm d}{{\rm d}t}
}y \left( t \right) =-b_{{1}}x \left( t \right) -a_{{2}}y \left( t
 \right)
\end{equation}

де $\Theta(t, s) = A (t) \Theta (t, s)$, $\Theta (s, s) = I$

Тут у позначеннях постанови задачі:
\begin{equation}
A(t) = A = 
\begin{bmatrix}
    -a_{1} & -b_{2}\\
    -b_{1} & -a_{2}
\end{bmatrix}
\end{equation}
\begin{equation}
B(t) = B = 
\begin{bmatrix}
    1 & 0\\
    0 & 1
\end{bmatrix}
\end{equation}

Очислемо матрицю $\Theta$:

\begin{equation}
\Theta \left( t,s \right) = \left[ \begin {array}{cc} {\frac {p_{{1}}{
{\rm e}^{q_{{2}} \left( s-t \right) }}+{{\rm e}^{q_{{1}} \left( s-t
 \right) }}p_{{2}}}{{k}^{2}}}&{\frac {b_{{2}} \left( -{{\rm e}^{q_{{2}
} \left( s-t \right) }}+{{\rm e}^{q_{{1}} \left( s-t \right) }}
 \right) }{{k}^{2}}}\\ \noalign{\medskip}{\frac {p_{{1}}p_{{2}}
 \left( -{{\rm e}^{q_{{2}} \left( s-t \right) }}+{{\rm e}^{q_{{1}}
 \left( s-t \right) }} \right) }{{k}^{2}b_{{2}}}}&{\frac {p_{{2}}{
{\rm e}^{q_{{2}} \left( s-t \right) }}+{{\rm e}^{q_{{1}} \left( s-t
 \right) }}p_{{1}}}{{k}^{2}}}\end {array} \right]
\end{equation}

Тут:

\begin{equation}
{k}=\sqrt {{a_{{1}}}^{2}-2\,a_{{1}}a_{{2}}+{a_{{2}}}^{2}+4\,b_{{1}}b_{{2}}}
\end{equation}
\begin{equation}
q_{{1}}=1/2\,{k}+1/2\,a_{{1}}+1/2\,a_{{2}}
\end{equation}
\begin{equation}
q_{{2}}=1/2\,{k}-1/2\,a_{{1}}-1/2\,a_{{2}}
\end{equation}
\begin{equation}
p_{{1}}=1/2\,{k}-1/2\,a_{{1}}+1/2\,a_{{2}}
\end{equation}
\begin{equation}
p_{{2}}=1/2\,{k}+1/2\,a_{{1}}-1/2\,a_{{2}}
\end{equation}

Знайдемо опорну функцію від $X(t, M_0)$:

\begin{equation}
c(X(t, M_0) , \psi) = c(M_0,\Theta^* (t, t_0) \psi) +\int_{t_0}^t c(U(s), B^*(s) \Theta^* (t, s) \psi)\mathrm{d}s
\end{equation}

\begin{equation}
c(M_0,\Theta^* (t, t_0) \psi) =
\begin{cases}
A_{{1}}N_{{1}}&0\leq A_{{1}}\cr 0&A_{{1}}<0\cr
\end{cases}
+
\begin{cases}
A_{{2}}N_{{2}}&0\leq A_{{2}}\cr 0&A_{{2}}<0\cr
\end{cases}
\end{equation}


\begin{equation}
A_{{1}}=A_{{1,1}}\psi_{{1}}-A_{{1,2}}\psi_{{2}}
\end{equation}

\begin{equation}
A_{{2}}=-A_{{2,1}}\psi_{{1}}+A_{{2,2}}\psi_{{2}}
\end{equation}

\begin{equation}
A_{{1,1}}={\frac {p_{{2}}{{\rm e}^{-Ta_{{1}}}}+{{\rm e}^{Ta_{{2}}}}p_{{1}}}{k}}
\end{equation}

\begin{equation}
A_{{1,2}}=-{\frac {p_{{1}}p_{{2}} \left( {{\rm e}^{-Ta_{{1}}}}-{
{\rm e}^{Ta_{{2}}}} \right) }{kb_{{2}}}}
\end{equation}

\begin{equation}
A_{{2,1}}=-{\frac {b_{{2}} \left( {{\rm e}^{-Ta_{{1}}}}-{{\rm e}^{Ta_{
{2}}}} \right) }{k}}
\end{equation}

\begin{equation}
A_{{2,2}}={\frac {{{\rm e}^{-Ta_{{1}}}}p_{{1}}+{{\rm e}^{Ta_{{2}}}}p_{
{2}}}{k}}
\end{equation}

\begin{equation}
c( U(s), B^*(s) \Theta^* (t, s) \psi)=|B_1|*h_1+|B_2|*h_2
\end{equation}

\begin{equation}
B_{{1}}=B_{{1,1}}\psi_{{1}}-B_{{1,2}}\psi_{{2}}
\end{equation}
\begin{equation}
B_{{2}}=-B_{{2,1}}\psi_{{1}}+B_{{2,2}}\psi_{{2}}
\end{equation}
\begin{equation}
B_{{1,1}}={\frac {{{\rm e}^{a_{{2}} \left( -s+T \right) }}p_{{1}}+{
{\rm e}^{-a_{{1}} \left( -s+T \right) }}p_{{2}}}{k}}
\end{equation}
\begin{equation}
B_{{1,2}}=-{\frac {p_{{1}}p_{{2}} \left( -{{\rm e}^{a_{{2}} \left( -s+
T \right) }}+{{\rm e}^{-a_{{1}} \left( -s+T \right) }} \right) }{kb_{{
2}}}}
\end{equation}
\begin{equation}
B_{{2,1}}={\frac {b_{{2}} \left( {{\rm e}^{a_{{2}} \left( -s+T
 \right) }}-{{\rm e}^{-a_{{1}} \left( -s+T \right) }} \right) }{k}}
\end{equation}
\begin{equation}
B_{{2,2}}={\frac {{{\rm e}^{a_{{2}} \left( -s+T \right) }}p_{{2}}+{
{\rm e}^{-a_{{1}} \left( -s+T \right) }}p_{{1}}}{k}}
\end{equation}

Тоді опорна функція від множини досяжності:
\begin{equation}
c(X(t, M_0) , \psi) =
\begin{cases}
A_{{1}}N_{{1}}&0\leq A_{{1}}\cr 0&A_{{1}}<0\cr
\end{cases}
+
\begin{cases}
A_{{2}}N_{{2}}&0\leq A_{{2}}\cr 0&A_{{2}}<0\cr
\end{cases}
+h_1\cdot\int_{t_0}^t |B_1(s)|\mathrm{d}s +h_2\cdot\int_{t_0}^t|B_2(s)| \mathrm{d}s
\end{equation}

Складність у подальшому обчисленні представляють інтеграли: $\int_{t_0}^t |B_1(s)|\mathrm{d}s$ та $\int_{t_0}^t|B_2(s)| \mathrm{d}s$

Зауважимо, що всі чисельні змінні додатні.

\section{Опис алгоритму}

\begin{equation}
{k}=\sqrt {{a_{{1}}}^{2}-2\,a_{{1}}a_{{2}}+{a_{{2}}}^{2}+4\,b_{{1}}b_{{2}}}
\end{equation}
\begin{equation}
q_{{1}}=1/2\,{k}+1/2\,a_{{1}}+1/2\,a_{{2}}
\end{equation}
\begin{equation}
q_{{2}}=1/2\,{k}-1/2\,a_{{1}}-1/2\,a_{{2}}
\end{equation}
\begin{equation}
p_{{1}}=1/2\,{k}-1/2\,a_{{1}}+1/2\,a_{{2}}
\end{equation}
\begin{equation}
p_{{2}}=1/2\,{k}+1/2\,a_{{1}}-1/2\,a_{{2}}
\end{equation}

%\begin{equation}
%A_{{1}}=A_{{1,1}}\psi_{{1}}-A_{{1,2}}\psi_{{2}}
%\end{equation}

%\begin{equation}
%A_{{2}}=-A_{{2,1}}\psi_{{1}}+A_{{2,2}}\psi_{{2}}
%\end{equation}

\begin{equation}
A_{{1,1}}={\frac {p_{{2}}{{\rm e}^{-Ta_{{1}}}}+{{\rm e}^{Ta_{{2}}}}p_{{1}}}{k}}
\end{equation}

\begin{equation}
A_{{1,2}}=-{\frac {p_{{1}}p_{{2}} \left( {{\rm e}^{-Ta_{{1}}}}-{
{\rm e}^{Ta_{{2}}}} \right) }{kb_{{2}}}}
\end{equation}

\begin{equation}
A_{{2,1}}=-{\frac {b_{{2}} \left( {{\rm e}^{-Ta_{{1}}}}-{{\rm e}^{Ta_{
{2}}}} \right) }{k}}
\end{equation}

\begin{equation}
A_{{2,2}}={\frac {{{\rm e}^{-Ta_{{1}}}}p_{{1}}+{{\rm e}^{Ta_{{2}}}}p_{
{2}}}{k}}
\end{equation}



%\begin{equation}
%B_{{1}}=B_{{1,1}}\psi_{{1}}-B_{{1,2}}\psi_{{2}}
%\end{equation}
%\begin{equation}
%B_{{2}}=-B_{{2,1}}\psi_{{1}}+B_{{2,2}}\psi_{{2}}
%\end{equation}
\begin{equation}
B_{{1,1}}={\frac {{{\rm e}^{a_{{2}} \left( -s+T \right) }}p_{{1}}+{
{\rm e}^{-a_{{1}} \left( -s+T \right) }}p_{{2}}}{k}}
\end{equation}
\begin{equation}
B_{{1,2}}=-{\frac {p_{{1}}p_{{2}} \left( -{{\rm e}^{a_{{2}} \left( -s+
T \right) }}+{{\rm e}^{-a_{{1}} \left( -s+T \right) }} \right) }{kb_{{
2}}}}
\end{equation}
\begin{equation}
B_{{2,1}}={\frac {b_{{2}} \left( {{\rm e}^{a_{{2}} \left( -s+T
 \right) }}-{{\rm e}^{-a_{{1}} \left( -s+T \right) }} \right) }{k}}
\end{equation}
\begin{equation}
B_{{2,2}}={\frac {{{\rm e}^{a_{{2}} \left( -s+T \right) }}p_{{2}}+{
{\rm e}^{-a_{{1}} \left( -s+T \right) }}p_{{1}}}{k}}
\end{equation}
\section{Опис програми}

Програма послідовно рахує коефіцієнти

\section{Опис обчислювальних експериментiв}

Для моделюваняя множини досяжності вирішувалась поставленна задача з константними керуючими функціями. Як результат був отриманий паралелограм. Задачі вирішувалися окремо з дискретним кроком. Точки досяжності візуалізуються на площині (див. додаток).

\section{Висновки}

Досить складно обрахувати інтеграл від модуля функції довільної структури.

Розв'язок при $ lim t -> \inf $ не залежить від початкового стану.

Якщо вирішувати задачу тільки коли змінні додатні, це не тривіальна задача та не має аналітичного розв'язку.

\section{Лiтература}

\end{document}
