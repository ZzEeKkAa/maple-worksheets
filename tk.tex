%\documentclass{article}
%\usepackage[ukrainian,english]{babel}
%\usepackage[utf8]{inputenc}
%\usepackage[T2A]{fontenc}
%\usepackage{amsmath}

%\title{Експериментальна робота №5 \\
%з курсу „Теорiя керування” \\
%студента 3-го курсу групи ОМ \\
%Гаврилка Євгенія}
%\date{2017-05-26}
%\author{Гаврилко Євгеній}

\documentclass[fontsize=14pt,DIV=1,a4paper]{scrartcl} 
\usepackage[utf8]{inputenc}
\usepackage{cmap} % для кодировки шрифтов в pdf
\usepackage[T2A]{fontenc}
%\usepackage{fontspec}
%\usepackage{polyglossia}
%\setmainfont{Times New Roman}
%\newfontfamily\cyrillicfont{Times New Roman}
%\usepackage{mathptmx}
%\renewcommand{\rmdefault}{ftm} % Times New Roman
\linespread{1.3} % полуторный интервал
\frenchspacing
\usepackage[english,russian,ukrainian]{babel}
\usepackage{indentfirst}
\usepackage{misccorr}
\usepackage{graphicx}
\usepackage[left=3cm,right=1.5cm,top=2cm,bottom=2cm,includefoot]{geometry}
\usepackage{graphicx}% для вставки картинок
%\graphicspath{ {/home/zeka/github.com/ZzEeKkAa/fuzzy-logic/} }
\graphicspath{ {/home/zeka/cursach/} }

\usepackage{amssymb,amsfonts,amsmath,amsthm} % математические дополнения от АМС
\usepackage{indentfirst} % отделять первую строку раздела абзацным отступом тоже
\usepackage[usenames,dvipsnames]{color} % названия цветов
\usepackage{makecell}
\usepackage{multirow} % улучшенное форматирование таблиц
\usepackage{ulem} % подчеркивания
\usepackage{gensymb}
\usepackage{hyperref}
\hypersetup{
    colorlinks=true,
    linkcolor=black,
    filecolor=black,      
    urlcolor=black,
}

\makeatletter
\DeclareOldFontCommand{\rm}{\normalfont\rmfamily}{\mathrm}
\DeclareOldFontCommand{\sf}{\normalfont\sffamily}{\mathsf}
\DeclareOldFontCommand{\tt}{\normalfont\ttfamily}{\mathtt}
\DeclareOldFontCommand{\bf}{\normalfont\bfseries}{\mathbf}
\DeclareOldFontCommand{\it}{\normalfont\itshape}{\mathit}
\DeclareOldFontCommand{\sl}{\normalfont\slshape}{\@nomath\sl}
\DeclareOldFontCommand{\sc}{\normalfont\scshape}{\@nomath\sc}
\makeatother


\begin{document}

\begin{titlepage}
		\begin{center}
			\large
			\textbf{КИЇВСЬКИЙ НАЦІОНАЛЬНИЙ УНІВЕРСИТЕТ \\
			ІМЕНІ ТАРАСА ШЕВЧЕНКА}

			\textbf{Кафедра обчислювальної математики}
			
			\vspace{6.0cm}

			\textsc{
			Експериментальна робота №5 \\ з курсу „Теорiя керування”}\\
			На тему:

			\textbf{ Модель бойових дiй двох армiй}
			\bigskip
		\end{center}
		\vfill

		\hfill
		\begin{minipage}{0.45\textwidth}
			студента 3-го курсу групи ОМ \\
			Гаврилка Євгенія
		\end{minipage}%
		\vfill

		\begin{center}
			Київ – 2017
		\end{center}
	\end{titlepage}
	
	\tableofcontents

%\maketitle
\newpage
\section{Опис математичної моделi i постановка задачi}
\subsection{Модель бойових дiй двох армiй}
Нехай в протиборствi беруть участь як регулярнi армiї, так i парти-
занськi об’єднання. Головною характеристикою суперникiв є чисельнiсть
сторiн $N_1(t) \geq 0$, $N_2(t) \geq 0$. У випадку дiй мiж регулярними частинами
динамiка їх чисельностi визначається факторами:
1). Швидкiсть зменшення особового складу за причинами, не зв’я-
заними безпосередньо з бойовими дiями: хвороби, травми, дезертирство;
2). Темп втрат обумовлений бойовими дiями; 3). Швидкiсть надання пiд-
крiплення, що вважається деякою функцiєю вiд часу.
При цих припущеннях для $N_1 (t)$, $N_2 (t)$ отримуємо систему диферен-
цiальних рiвнянь

\begin{equation}
{\frac {\rm d}{{\rm d}t}}N_1(t) = -a_1(t) N_1(t) - b_2(t) N_2(t) + g_1(t)
\end{equation}
\begin{equation}
{\frac {\rm d}{{\rm d}t}}N_2(t) = -b_1(t) N_1(t) - a_2(t) N_2(t) + g_2(t)
\end{equation}

Тут неперервнi функцiї $a_i(t) \geq  0$ визначають швидкiсть втрат у силу
причин, що не пов’язанi з бойовими дiями, неперервнi функцiї $b_i(t) \geq  0$
показують темпи втрат через бойовi дiї супротивника, неперервнi фун-
кцiї $g_i(t)$ показують темп надання пiдкрiплення (модель Ланчестера).

\subsection{Постановка задачі}
Розглядається система керування (1), (2), в якiй $g_1 (t), g_2(t)$ - функцiї керування,
$t \in [0, T]$, $a_1 (t) = a_1$, $a_2(t) = a_2$, $b_1 (t) = b_1$, $b_2 (t) = b_2$ - додатнi константи.

Побудувати множину досяжностi в момент $T > 0$ за умови, що обме-
ження на початковий стан задається нерiвностями

\begin{equation}
0\leq N_1\leq \alpha 
\end{equation}
\begin{equation}
0\leq N_2\leq \beta
\end{equation}
а обмеження на керування мають вигляд
\begin{equation}
-h_1\leq g_1(t)\leq h_1
\end{equation}
\begin{equation}
-h_2\leq g_2(t)\leq h_2
\end{equation}
де $\alpha, \beta, h1, h2$ – додатнi константи.
\section{Математичні викладки}

З теорії відомо, що система (1), (2) за умов (5), (6) є стандарною задачею множини досяжностi лiнiйної системи керування, яка може бути обчислена за формулою

\begin{equation}
X(t, M_0) = \Theta (t, t_0) M_0 +\int_{t_0}^t \Theta (t, s) B(s)U(s)\mathrm{d}s
\end{equation}
 
\begin{equation}
{\frac {\rm d}{{\rm d}t}}x \left( t \right) =-a_{{1}}x
 \left( t \right) -b_{{2}}y \left( t \right)
\end{equation}
\begin{equation}
 {\frac {\rm d}{{\rm d}t}
}y \left( t \right) =-b_{{1}}x \left( t \right) -a_{{2}}y \left( t
 \right)
\end{equation}

де $\Theta(t, s) = A (t) \Theta (t, s)$, $\Theta (s, s) = I$

Тут у позначеннях постанови задачі:
\begin{equation}
A(t) = A = 
\begin{bmatrix}
    -a_{1} & -b_{2}\\
    -b_{1} & -a_{2}
\end{bmatrix}
\end{equation}
\begin{equation}
B(t) = B = 
\begin{bmatrix}
    1 & 0\\
    0 & 1
\end{bmatrix}
\end{equation}

Очислемо матрицю $\Theta$:

\begin{equation}
\Theta \left( t,s \right) = \left[ \begin {array}{cc} {\frac {p_{{1}}{
{\rm e}^{q_{{2}} \left( s-t \right) }}+{{\rm e}^{q_{{1}} \left( s-t
 \right) }}p_{{2}}}{{k}^{2}}}&{\frac {b_{{2}} \left( -{{\rm e}^{q_{{2}
} \left( s-t \right) }}+{{\rm e}^{q_{{1}} \left( s-t \right) }}
 \right) }{{k}^{2}}}\\ \noalign{\medskip}{\frac {p_{{1}}p_{{2}}
 \left( -{{\rm e}^{q_{{2}} \left( s-t \right) }}+{{\rm e}^{q_{{1}}
 \left( s-t \right) }} \right) }{{k}^{2}b_{{2}}}}&{\frac {p_{{2}}{
{\rm e}^{q_{{2}} \left( s-t \right) }}+{{\rm e}^{q_{{1}} \left( s-t
 \right) }}p_{{1}}}{{k}^{2}}}\end {array} \right]
\end{equation}

Тут:

\begin{equation}
{k}=\sqrt {{a_{{1}}}^{2}-2\,a_{{1}}a_{{2}}+{a_{{2}}}^{2}+4\,b_{{1}}b_{{2}}}
\end{equation}
\begin{equation}
q_{{1}}=1/2\,{k}+1/2\,a_{{1}}+1/2\,a_{{2}}
\end{equation}
\begin{equation}
q_{{2}}=1/2\,{k}-1/2\,a_{{1}}-1/2\,a_{{2}}
\end{equation}
\begin{equation}
p_{{1}}=1/2\,{k}-1/2\,a_{{1}}+1/2\,a_{{2}}
\end{equation}
\begin{equation}
p_{{2}}=1/2\,{k}+1/2\,a_{{1}}-1/2\,a_{{2}}
\end{equation}

Зауважимо, що діагональні елементи фундаментальної матриці додадтні, а інші - від'ємні

Знайдемо множину досяжності $X(t, M_0)$:

\begin{equation}
X(t, M_0) = \bigcup_{x \in M_0} \bigcup_{u(t) \in U(t)} \Theta (T, 0)\cdot x_0 +\int_{0}^T \Theta (T, s) \cdot B(s)\cdot u(s) \mathrm{d}s
\end{equation}

Враховуючи, що кожен доданок залежить лише від однієї змінної об'єднання та $B(s)=\left[ \begin {array}{cc}
{1}&{0}\\ \noalign{\medskip}
{0}&{1}\end {array} \right]$:


\begin{equation}
X(T, M_0) = \bigcup_{x \in M_0} \Theta (T, 0)\cdot x_0 + \bigcup_{u(t) \in U(t)} \int_{t_0}^T \Theta (T, s) \cdot u(s) \mathrm{d}s
\end{equation}

Розглянемо другий доданок: 
\begin{equation}
	\bigcup_{u(t) \in U(t)} \int_{0}^T \Theta (T, s) \cdot u(s) \mathrm{d}s = 
	\bigcup_{|u_i(t)| \leq h_i} \int_{0}^T \left[ \begin {array}{cc}
	{B_{1,1}(s)}&{B_{1,2}(s)}\\ \noalign{\medskip}
	{B_{2,1}(s)}&{B_{2,2}(s)}\end {array} \right]\cdot \left[ \begin {array}{cc}
	{u_{1}(s)}\\ \noalign{\medskip}
	{u_{2}(s)}\end {array} \right] \mathrm{d}s
\end{equation}

Доведемо, що наклавши на умову $u_{i}$ що вона константа ми отримаэмо ту саму множину.

Не обмежуючи загальності вважаємо, що $h_i=1$. Тоді розкрівши добуток і розглядаючи пари доданків з фіксованим i, маэмо 2 інтеграли (додатній та відємний на всьому проміжку) $u_i(s)$ - задають лише на скільки зтиснути інтеграл. Таким чинном відношення пложин не змінюється, отже існує константа, така, що інтеграли мають таке саме значення.

Тому можемо інтегрувати $B_{i,j}(s)$ окремо.

\begin{equation}
X(T, M_0) = 
\bigcup_{0 \leq|x_{i,0}| \leq N_i}
\left[ \begin {array}{cc}
	{A_{1,1}}&{A_{1,2}}\\ \noalign{\medskip}
	{A_{2,1}}&{A_{2,2}}\end {array} \right] \cdot
\left[ \begin {array}{cc}
	{x_{1,0}}\\ \noalign{\medskip}
	{x_{2,0}}\end {array} \right]
+ 
\bigcup_{|u_i| \leq h_i}
\left[ \begin {array}{cc}
	{B_{1,1}}&{B_{1,2}}\\ \noalign{\medskip}
	{B_{2,1}}&{B_{2,2}}\end {array} \right] \cdot
\left[ \begin {array}{cc}
	{u_{1,0}}\\ \noalign{\medskip}
	{u_{2,0}}\end {array} \right]
\end{equation}


\begin{equation}
A_{{1,1}}={\frac {p_{{1}}{{\rm e}^{q_{{2}}T}}+p_{{2}}{{\rm e}
^{-q_{{1}}T}}}{k}}
\end{equation}
\begin{equation}
A_{{1,2}}={\frac {b_{{2}} \left( -{{\rm e}^{q_{{2}}
T}}+{{\rm e}^{-q_{{1}}T}} \right) }{k}}
\end{equation}
\begin{equation}
A_{{2,1}}=-{\frac {p_{{1}}p_{{
2}} \left( {{\rm e}^{q_{{2}}T}}-{{\rm e}^{-q_{{1}}T}} \right) }{kb_{{2
}}}}
\end{equation}
\begin{equation}
A_{{2,2}}={\frac {{{\rm e}^{q_{{2}}T}}p_{{2}}+{{\rm e}^{-q_{{1}}T
}}p_{{1}}}{k}}
\end{equation}
\begin{equation}
B_{{1,1}}={\frac {p_{{1}}{{\rm e}^{q_{{2}}T}}q_{{1}}-p_
{{2}}{{\rm e}^{-q_{{1}}T}}q_{{2}}-p_{{1}}q_{{1}}+p_{{2}}q_{{2}}}{kq_{{
2}}q_{{1}}}}
\end{equation}
\begin{equation}
B_{{1,2}}=-{\frac {b_{{2}} \left( {{\rm e}^{q_{{2}}T}}q_{
{1}}+{{\rm e}^{-q_{{1}}T}}q_{{2}}-q_{{1}}-q_{{2}} \right) }{kq_{{2}}q_
{{1}}}}
\end{equation}
\begin{equation}
B_{{2,1}}=-{\frac {p_{{1}}p_{{2}} \left( {{\rm e}^{q_{{2}}T}}q
_{{1}}+{{\rm e}^{-q_{{1}}T}}q_{{2}}-q_{{1}}-q_{{2}} \right) }{kb_{{2}}
q_{{2}}q_{{1}}}}
\end{equation}
\begin{equation}
B_{{2,2}}={\frac {p_{{2}}{{\rm e}^{q_{{2}}T}}q_{{1}}-
p_{{1}}{{\rm e}^{-q_{{1}}T}}q_{{2}}+p_{{1}}q_{{2}}-p_{{2}}q_{{1}}}{kq_
{{2}}q_{{1}}}}
\end{equation}

Як бачимо, ми маємо сумму 2х параллелограмів. Враховуючи що сумма опуклих є опуклою, нам залишається знайти опуклу оболонку точок, що утворюються сумами вершин паралеллограмів. Їх всього 16.

\section{Опис алгоритму}

\begin{equation}
{k}=\sqrt {{a_{{1}}}^{2}-2\,a_{{1}}a_{{2}}+{a_{{2}}}^{2}+4\,b_{{1}}b_{{2}}}
\end{equation}
\begin{equation}
q_{{1}}=1/2\,{k}+1/2\,a_{{1}}+1/2\,a_{{2}}
\end{equation}
\begin{equation}
q_{{2}}=1/2\,{k}-1/2\,a_{{1}}-1/2\,a_{{2}}
\end{equation}
\begin{equation}
p_{{1}}=1/2\,{k}-1/2\,a_{{1}}+1/2\,a_{{2}}
\end{equation}
\begin{equation}
p_{{2}}=1/2\,{k}+1/2\,a_{{1}}-1/2\,a_{{2}}
\end{equation}

\begin{equation}
A_{{1,1}}={\frac {p_{{1}}{{\rm e}^{q_{{2}}T}}+p_{{2}}{{\rm e}
^{-q_{{1}}T}}}{k}}
\end{equation}
\begin{equation}
A_{{1,2}}={\frac {b_{{2}} \left( -{{\rm e}^{q_{{2}}
T}}+{{\rm e}^{-q_{{1}}T}} \right) }{k}}
\end{equation}
\begin{equation}
A_{{2,1}}=-{\frac {p_{{1}}p_{{
2}} \left( {{\rm e}^{q_{{2}}T}}-{{\rm e}^{-q_{{1}}T}} \right) }{kb_{{2
}}}}
\end{equation}
\begin{equation}
A_{{2,2}}={\frac {{{\rm e}^{q_{{2}}T}}p_{{2}}+{{\rm e}^{-q_{{1}}T
}}p_{{1}}}{k}}
\end{equation}
\begin{equation}
B_{{1,1}}={\frac {p_{{1}}{{\rm e}^{q_{{2}}T}}q_{{1}}-p_
{{2}}{{\rm e}^{-q_{{1}}T}}q_{{2}}-p_{{1}}q_{{1}}+p_{{2}}q_{{2}}}{kq_{{
2}}q_{{1}}}}
\end{equation}
\begin{equation}
B_{{1,2}}=-{\frac {b_{{2}} \left( {{\rm e}^{q_{{2}}T}}q_{
{1}}+{{\rm e}^{-q_{{1}}T}}q_{{2}}-q_{{1}}-q_{{2}} \right) }{kq_{{2}}q_
{{1}}}}
\end{equation}
\begin{equation}
B_{{2,1}}=-{\frac {p_{{1}}p_{{2}} \left( {{\rm e}^{q_{{2}}T}}q
_{{1}}+{{\rm e}^{-q_{{1}}T}}q_{{2}}-q_{{1}}-q_{{2}} \right) }{kb_{{2}}
q_{{2}}q_{{1}}}}
\end{equation}
\begin{equation}
B_{{2,2}}={\frac {p_{{2}}{{\rm e}^{q_{{2}}T}}q_{{1}}-
p_{{1}}{{\rm e}^{-q_{{1}}T}}q_{{2}}+p_{{1}}q_{{2}}-p_{{2}}q_{{1}}}{kq_
{{2}}q_{{1}}}}
\end{equation}

Знайти 4 точки, для $x=0,N_1$; $y=0,N_2$:
\begin{equation}
[A_{1,1}\cdot x+A_{1,2}\cdot y,A_{2,1}\cdot x+A_{2,2}\cdot y]
\end{equation}
Знайти 4 точки, для $x=-h_1,h_1$; $y=-h_2,h_2$:
\begin{equation}
[B_{1,1}\cdot x+B_{1,2}\cdot y,B_{2,1}\cdot x+B_{2,2}\cdot y]
\end{equation}

Попарно просумувати і отримати 16 точок.
Знайти їх опуклу оболонку, що складається з 8 точок.

\section{Опис програми}

Програма послідовно рахує коефіцієнти. Опуклу оболонку за алгоритмом.

\section{Опис обчислювальних експериментiв}

Для моделюваняя множини досяжності вирішувалась поставленна задача з константними керуючими функціями та періодичними. Як результат був отриманий 8 кутник. Задачі вирішувалися окремо з дискретним кроком. Точки досяжності візуалізуються на площині (див. додаток).

\section{Висновки}

Керування однаково змінює множину досяжності не залежно від початкового стану.

Сумма 2-х параллелограмів - 8 кутник з параллельними протилежними сторонами.
	
	\newpage
	\addcontentsline{toc}{section}{Література}
	\begin{thebibliography}{9}
		\bibitem{control_theory}
		Лекцiї з теорiї керування. Пiчкур В.В.
		\bibitem{convex_hull}
		\url{https://en.wikibooks.org/wiki/Algorithm_Implementation/Geometry/Convex_hull/Monotone_chain}
		\bibitem{golang}
		\url{https://golang.org/}
	\end{thebibliography}
\end{document}
